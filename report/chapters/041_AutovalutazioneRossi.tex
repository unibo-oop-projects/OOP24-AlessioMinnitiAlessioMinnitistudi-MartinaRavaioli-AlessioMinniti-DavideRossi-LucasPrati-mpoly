\subsection{Davide Rossi}
Nell'ambito del progetto mi sono occupato principalmente di realizzare la parte della banca compresa della gestione del denaro degli utenti e
dei contratti di proprietà. All'interno del gruppo ho assunto un po' il ruolo di una figura di riferimento. 
Seppure io mi sia sempre occupato esclusivamente della mia parte spesso gli altri membri si sono rivolti a me per consigli.
Personalmente ritengo di aver affrontato il lavoro di gruppo con professionalità. 
Ho sempre cercato di orientare le mie scelte progettuali favorendo riusabilità e estendibilità, 
sfruttando le conoscenze apprese nel corso per identificare i punti di debolezza del mio codice e produrre nuove soluzioni più congeniali.
Ho cercato di immedesimarmi nella realizzazione di un’applicazione in un contesto lavorativo, 
e quindi progettare le mie parti in modo che fossero facili da usare e modificare. 
A volte tuttavia tendo a prediligere soluzioni eccessivamente complesse, 
realizzando componenti che fanno ben più di ciò che è richiesto e ho notato che 
questo si è rivelato un problema nella maggior parte dei casi; perché seppure lo facessi per favorire 
l’espandibilità e coprire ogni caso d'uso o modifica futura spesso semplicemente il risultato era un sistema complesso e difficile 
da utilizzare. 
Se si dovesse portare avanti il progetto spenderei maggiori energie per rivedere 
alcune parti dell’architettura del model. Credo che parte della debolezza 
dell’architettura attuale sia dovuta al fatto che i componenti principali hanno 
necessità di scambiare dati tra di loro e per fare ciò  sono state prodotte delle 
soluzioni accettabili che tuttavia hanno margine di miglioramento. 
In particolare, per quanto riguarda la mia parte, rivaluterei la soluzione dell'interfaccia BankState per l' interazione con
TurnationManager.
