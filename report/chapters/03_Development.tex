\section{Testing automatizzato}
Per il testing automatico dell'architettura del progetto abbiamo utilizzato la libreria di JUnit che ci ha consentito di testare le parti critiche del progetto in maniera ottimale.
Per quanto riguarda la parte monetaria e di transazioni del gioco sono stati effettuati i seguenti test:\newline
\begin{itemize}
    \item \textbf{BankTest}: test delle principali funzioni dell'interfaccia Bank verificandone il corretto comportamento e risposta agli errori anche in diverse situazioni
    \item \textbf{BankStateTest}: test dell'interfaccia BankState avvenuto congiuntamente con il corrispettivo oggetto Bank, controllando il funzionamento dei metodi per il controllo e l'alterazione dello stato della banca 
    \item \textbf{TitleDeedTest}: sono stati realizzati dei test per entrambe le versioni di title deed presenti nel gioco, con o senza case. Sono state testate le funzioni per i vari costi in particolare \texttt{getRent}, verificando che restituisse l'affitto corretto in base alla situazione, e il funzionamento dei metodi per assegnare o revocare il possesso di un contratto per un giocatore
    \item \textbf{RentOptionFactoryTest}: sono stati testati i metodi della factory verificando che le \texttt{RentOption} create avessero il prezzo giusto e controllando che i criteri di applicabilità fossero corretti.
    \item \textbf{SimpleBankAccountTest}: si è verificato il funzionamento dei metodi per il prelievo e il deposito di denaro e per verificare se il \texttt{BankAccount} è ancora in uno stato valido per la continuazione del gioco.  
\end{itemize}
Per quanto riguarda la parte di Board con le caselle per le proprietà, le caselle speciali e imprevisti sono stati effettuati i seguienti test:\newline
\begin{itemize}
    \item \textbf{BaseAndComplexCommandFactoryTest}: si è verificato il corretto funzionamento dei vari comandi base creati e dei comandi complessi composti da questi ultimi.
    \item \textbf{BaseAndComplexInterpreterTest}: si è verificato il corretto funzionamento della creazione tramite interpreti dei comandi testati precedentemente.
    \item \textbf{DeckTest}: si è verificato il corretto funzionamento della creazione del deck di carte dato un file dove sono scritti, secondo le regole stabilite dagli interpreti, gli effetti delle varie carte
    \item \textbf{ParserOnSpaceTest}: si è verificato il corretto funzionamento dei vari parser usati per separare le varie parti del file da interpretare
    \item \textbf{FactoryTest}: si è verificato il corretto funzionamento della factory di caselle speciali e dei loro effetti. 
    \item \textbf{SpecialPropertyTest}: si è verificato il corretto funzionamento delle proprietà speciali e la corretta modifica dell' affito.
    \item \textbf{BoardTest}: Test per il corretto funzionamento della board. Viene controllata la gestione delle pawns e delle tiles sia property che special, inoltre viene testato anche la gestione per l'aggiunta delle case e degli alberghi nelle proprietà.
    \item \textbf{PropertyTest}: Test per controllare il funzionamento delle proprietà. Controlla che solo le proprietà giuste possano costruire case e hotel e che possano farlo solo sottostando a determinate condizioni.
\end{itemize}
Per quanto riguarda la parte di gestione dei turni e dei player sono stati effettuati i seguenti test:\newline
\begin{itemize}
    \item \textbf{TurnationManagerTest}: Test per il corretto funzionamento della gestione del turnation manager, viene testato il corretto funzionamento dei turni verificando la successione dei player e del controllo dei loro stati (park, prison)
\end{itemize}
Per quanto riguarda la parte di caricamento delle risorse, sono stati effettuati i seguenti test:\newline
\begin{itemize}
    \item \textbf{UseFileTxtImplTest}: si è verificata la corretta ricerca dei file di testo e la corretta gestione delle eccezioni e la segnalazioni di eventuali errori.
    \item \textbf{UseFileJsonImplTest}: si è verificata la corretta ricerca del file delle tessere di gioco e la corretta deserializzazione in un DTO temporaneo.
    \item \textbf{UseConfigurationFileImplTest}: è stato verificato il corretto caricamento di una risorsa di configurazione attendibile e priva di errori. Si è verificato anche il fall-back alla configurazione di default in caso di errori durante il caricamento della risorsa.
    \item \textbf{ConfigurationTest}: sono stati verificati tutti i parametri della configurazione e i relativi casi limite. Sono state anche verificate configurazioni inconsistenti e la corretta validazione di una configurazione valida.
\end{itemize}
Per quanto riguarda i giocatori, la creazione dei conti correnti e la creazione di tessere e contratti di proprietà
\section{Note di sviluppo}