\chapter{Design}

\section{Architettura}
L’architettura di MPoly è basata sul pattern architetturale MVC. 
Si è scelto di implementare l’architettura nella sua forma classica in quanto il gioco 
ha una modalità di interazione prettamente statica: l’utente interagisce con la view, 
facendo scaturire un evento; a seguito di questo evento si eseguono una serie di operazioni 
specifiche per la sua gestione con conseguente aggiornamento della view, che mostra all’utente che cosa è successo.   
È stato predisposto un controller principale denominato GameManager. 
Questo implementa il pattern observer in quanto si mette in ascolto degli eventi che vengono catturati dalla view. 
Nel momento in cui viene notificato di un evento il controller interroga il model, 
nello specifico facendo delle chiamate a Board, che si occupa di gestire la struttura del tabellone,
TurnationManager, che si occupa di gestire l’avanzamento della partita, 
e TransactionManager, per gestire lo scambio di denaro. 
Questi sono i punti d’ entrata del model e offrono delle primitive che incapsulano la logica di funzionamento 
delle principali azioni che si possono compiere durante il gioco. Queste azioni sono caratteristiche del 
gioco stesso Monopoly. 
Con questa architettura il modello è perfettamente scorporabile e utilizzabile per costruire 
un software diverso con lo stesso principio di funzionamento. 
Una volta finita l’interrogazione del model il GameManager si occuperà di chiamare 
delle funzioni sulla view che aggiornano il contenuto di quest’ultima. 
Il controller agisce su model e view mediante delle interfacce che sono completamente indipendenti 
dall’implementazione di quest’ultimi. 
Questo fa sì che la modalità di implementazione della view non determini cambiamenti sul 
controller o sul model in alcun caso. 
Il software prevede anche un menù iniziale di configurazione della partita. 
Questo menù è a sua volta costituito da una sua architettura MVC più ridotta, 
modellata tenendo in mente gli stessi principi descritti sovrastante. 
L’entità GameBuilder è colei che si occupa di creare poi l’MVC principale del software e avviare effettivamente il gioco.

\begin{figure}[H]
    \centering
    \includegraphics[width=0.5\textheight]{img/architecture_diagram.png}
    \caption{Schema UML dell'analisi del problema, con rappresentate le entità principali ed i rapporti fra loro}
	\label{img:architecture_diagram}
\end{figure}
